\let\negmedspace\undefined
\let\negthickspace\undefined
\documentclass[journal,12pt,twocolumn]{IEEEtran}
\usepackage{gensymb}
\usepackage{amssymb}
\usepackage[cmex10]{amsmath}
\usepackage{amsmath}
\usepackage{amsthm}
\usepackage[export]{adjustbox}
\usepackage{bm}
\usepackage{longtable}
\usepackage{enumitem}
\usepackage{mathtools}
 \usepackage{tikz}
\usepackage[breaklinks=true]{hyperref}
\usepackage{listings}
\usepackage{color}                                            %%
\usepackage{array}                                            %%
\usepackage{longtable}                                        %%
\usepackage{calc}                                             %%
\usepackage{multirow}                                         %%
\usepackage{hhline}                                           %%
\usepackage{ifthen}                                           %%
\usepackage{lscape}     
\usepackage{multicol}
\usepackage{float}
% \usepackage{enumerate}
\DeclareMathOperator*{\Res}{Res}
\renewcommand\thesection{\arabic{section}}
\renewcommand\thesubsection{\thesection.\arabic{subsection}}
\renewcommand\thesubsubsection{\thesubsection.\arabic{subsubsection}}
\renewcommand\thesectiondis{\arabic{section}}
\renewcommand\thesubsectiondis{\thesectiondis.\arabic{subsection}}
\renewcommand\thesubsubsectiondis{\thesubsectiondis.\arabic{subsubsection}}
\hyphenation{op-tical net-works semi-conduc-tor}
\def\inputGnumericTable{}                                 %%
\lstset{
frame=single, 
breaklines=true,
columns=fullflexible
}
\begin{document}
\newtheorem{theorem}{Theorem}[section]
\newtheorem{problem}{Problem}
\newtheorem{proposition}{Proposition}[section]
\newtheorem{lemma}{Lemma}[section]
\newtheorem{corollary}[theorem]{Corollary}
\newtheorem{example}{Example}[section]
\newtheorem{definition}[problem]{Definition}
\newcommand{\BEQA}{\begin{eqnarray}}
\newcommand{\EEQA}{\end{eqnarray}}
\newcommand{\define}{\stackrel{\triangle}{=}}
\newcommand*\circled[1]{\tikz[baseline=(char.base)]{
    \node[shape=circle,draw,inner sep=2pt] (char) {#1};}}
\bibliographystyle{IEEEtran}
\providecommand{\mbf}{\mathbf}
\providecommand{\pr}[1]{\ensuremath{\Pr\left(#1\right)}}
\providecommand{\qfunc}[1]{\ensuremath{Q\left(#1\right)}}
\providecommand{\sbrak}[1]{\ensuremath{{}\left[#1\right]}}
\providecommand{\lsbrak}[1]{\ensuremath{{}\left[#1\right.}}
\providecommand{\rsbrak}[1]{\ensuremath{{}\left.#1\right]}}
\providecommand{\brak}[1]{\ensuremath{\left(#1\right)}}
\providecommand{\lbrak}[1]{\ensuremath{\left(#1\right.}}
\providecommand{\rbrak}[1]{\ensuremath{\left.#1\right)}}
\providecommand{\cbrak}[1]{\ensuremath{\left\{#1\right\}}}
\providecommand{\lcbrak}[1]{\ensuremath{\left\{#1\right.}}
\providecommand{\rcbrak}[1]{\ensuremath{\left.#1\right\}}}
\theoremstyle{remark}
\newtheorem{rem}{Remark}
\newcommand{\sgn}{\mathop{\mathrm{sgn}}}
\providecommand{\abs}[1]{\left\vert#1\right\vert}
\providecommand{\res}[1]{\Res\displaylimits_{#1}} 
\providecommand{\norm}[1]{\left\lVert#1\right\rVert}
%\providecommand{\norm}[1]{\lVert#1\rVert}
\providecommand{\mtx}[1]{\mathbf{#1}}
\providecommand{\mean}[1]{E\left[ #1 \right]}
\providecommand{\fourier}{\overset{\mathcal{F}}{ \rightleftharpoons}}
%\providecommand{\hilbert}{\overset{\mathcal{H}}{ \rightleftharpoons}}
\providecommand{\system}{\overset{\mathcal{H}}{ \longleftrightarrow}}
	%\newcommand{\solution}[2]{\textbf{Solution:}{#1}}
\newcommand{\solution}{\noindent \textbf{Solution: }}
\newcommand{\cosec}{\,\text{cosec}\,}
\providecommand{\dec}[2]{\ensuremath{\overset{#1}{\underset{#2}{\gtrless}}}}
\newcommand{\myvec}[1]{\ensuremath{\begin{pmatrix}#1\end{pmatrix}}}
\newcommand{\mydet}[1]{\ensuremath{\begin{vmatrix}#1\end{vmatrix}}}
\newcommand*{\permcomb}[4][0mu]{{{}^{#3}\mkern#1#2_{#4}}}
\newcommand*{\perm}[1][-3mu]{\permcomb[#1]{P}}
\newcommand*{\comb}[1][-1mu]{\permcomb[#1]{C}}
\numberwithin{equation}{subsection}
\makeatletter
\@addtoreset{figure}{problem}
\makeatother
\let\StandardTheFigure\thefigure
\let\vec\mathbf
\renewcommand{\thefigure}{\theproblem}
\def\putbox#1#2#3{\makebox[0in][l]{\makebox[#1][l]{}\raisebox{\baselineskip}[0in][0in]{\raisebox{#2}[0in][0in]{#3}}}}
     \def\rightbox#1{\makebox[0in][r]{#1}}
     \def\centbox#1{\makebox[0in]{#1}}
     \def\topbox#1{\raisebox{-\baselineskip}[0in][0in]{#1}}
     \def\midbox#1{\raisebox{-0.5\baselineskip}[0in][0in]{#1}}
\vspace{3cm}
\title{ASSIGNMENT}
\author{CS21BTECH11020 (Harsh Goyal)}

% make the title area
\maketitle
\newpage

\tableofcontents
\bigskip
\renewcommand{\thefigure}{\theenumi}
\renewcommand{\thetable}{\theenumi}
\renewcommand{\theequation}{\theenumi}


\section{Uniform Random Numbers}



Let $U$ be a uniform  random variable between 0 and 1
\begin{enumerate}[label=\thesection.\arabic*.,ref=\thesection.\theenumi]
\numberwithin{equation}{enumi}
\numberwithin{figure}{enumi}
\numberwithin{table}{enumi}
\item Generate $10^6$ samples of $U$ using a C program and save into a file called uni.dat .\\

\solution Download the file:

\begin{lstlisting}
$ wget https://raw.githubusercontent.com/galaxion-tech/AI1110/master/ass_manual/code/1.1.c
$ wget https://raw.githubusercontent.com/galaxion-tech/AI1110/master/ass_manual/code/header/coeffs.h
\end{lstlisting}

and compile and execute the C program using

\begin{lstlisting}
$ gcc 1.1.c -lm -Wall -g
$ ./a.out
\end{lstlisting}





\item Load the uni.dat file into python and plot the empirical CDF of $U$ using the samples in uni.dat. The CDF is defined as\\

\solution  The following code plots Fig. \ref{fig:uni_cdf}

\begin{lstlisting}
$ wget https://raw.githubusercontent.com/galaxion-tech/AI1110/master/ass_manual/code/1.2.py
\end{lstlisting}

It is executed with

\begin{lstlisting}
$ python3 1.2.py
\end{lstlisting}

\begin{equation}
         F_U(x) = Pr(U \leq x)
\end{equation}

    Graph of CDF is as follow:

    \begin{figure}[H]
        \centering
        \includegraphics[scale = 0.6]{./figs/uni_cdf} \label{fig:uni_cdf}
        \caption{CDF of U}
        \end{figure}

\item Find a theoretical expression for $F_U(x)$.\\

\solution Since We have,
    
    \begin{align}
    P_U(x) =  \begin{cases}
        1 & x \in (0,1) \\
        0 & otherwise
    \end{cases} 
    \end{align}

    on integrating for CDF we get,
   \begin{equation}
    F_U(x) = \int_{-\infty}^{x}P_U(t)dt
   \end{equation} 

   \begin{align}
    F_U(x) =  \begin{cases}
        \int_{-\infty}^{x}0 dx   & x \in (-\infty,0)\\
        %\int_{-\infty}^0 0 dx + 
        \int_{0}^x 1dx & x \in (0,1) \\
        %\int_{-\infty}^0 0 dx +
         \int_{0}^1 1dx & x \in (1,\infty)
 %+ \int_{1}^{x} 0 dx  & x \in (1,\infty)
    \end{cases} 
    \end{align}

    \begin{align}
    F_U(x) =  \begin{cases}
        0 & x \in (-\infty,0)\\
        x & x \in (0,1) \\
        1 & x \in (1,\infty)
    \end{cases} 
    \end{align}


\item Write a C program to find the mean and variance of U.\\

\solution download C program

\begin{lstlisting}
$ wget https://raw.githubusercontent.com/galaxion-tech/AI1110/master/ass_manual/code/1.4.c
        \end{lstlisting}
        and compiled and executed with
        \begin{lstlisting}
$ gcc 1.4.c -lm -Wall -g
$ ./a.out
        \end{lstlisting}
    \begin{align}
        E[U] &= 0.500007  \label{eq:1.4.1}\\
        \text{Var}[U] &= 0.083301 \label{eq:1.4.2}
    \end{align}
\item Verify your result theoretically given that 
    \begin{equation}
        E[U^k] = \int_{- \infty}^{\infty} x^k dF_U(x)
    \end{equation}
    \solution we have 
    \begin{align}
        E[U] &= \int_{-\infty}^{\infty} xdF_U(x) \\
        &= \int_{0}^1 x dx \\
        &=  0.5
    \end{align}
    From \eqref{eq:1.4.1}, we have, 
    \begin{equation}
        E[U] = 0.500007 \approx 0.5
    \end{equation}
    Similarly,
    \begin{align}
        \text{Var}[U]&=E[U^2]-(E[U])^2 \\
        &=\int_{-\infty}^{\infty}x^2dF_U(x) - 0.25\\
        &=\int_{0}^{1}x^2dx - 0.25\\
        &=0.3333...-0.25=0.083333...
    \end{align}
    From \eqref{eq:1.4.2},we get
    \begin{equation}
        \text{Var}[U] = 0.083301 \approx 0.08333..
    \end{equation}
    Hence Verified.
\end{enumerate}
\section{Central Limit Theorem}
\begin{enumerate}[label=\thesection.\arabic*.,ref=\thesection.\theenumi]
\numberwithin{equation}{enumi}
\numberwithin{figure}{enumi}
\numberwithin{table}{enumi}


\item Generate $10^6$ samples of the random variable
    \begin{equation}
        X=\sum_{i=1}^{12} U_i-6        
    \end{equation}
    using a C program, where $U_i, i = 1, 2,\ldots, 12$ are a set of independent uniform random variables between 0 and 1 and save in a file called gau.dat.\\
    
    
    \solution Download the file

    \begin{lstlisting}    
$ wget https://raw.githubusercontent.com/galaxion-tech/AI1110/master/ass_manual/code/2.1.c
    \end{lstlisting}

    Use coeffs.h from the prob1.1\\
    And run the code as:

    \begin{lstlisting}
$ gcc 2.1.c -lm -Wall -g
$ ./a.out
\end{lstlisting}

\item Load gau.dat in python and plot the empirical CDF of $X$ using the samples in gau.dat. What properties does a CDF have?\\

\solution
    Formula Used to calculate $F_X(x)$ is:
    \begin{align}
        F(x)&=1-Q(x)\\
            &=1-\frac{1}{2}erfc(\frac{x}{\sqrt{2}}) 
    \end{align}
    where,
    \begin{align}
        erfc(x)=1-erf(x) = 1-\frac{2}{\sqrt{\pi}}\int_0^z e^{-t^2}dt
    \end{align}
    Using $mpmath.erfc()$ function to calculate $erfc()$ in python code.
    The required python file can be downloaded using

    \begin{lstlisting}
 $ wget https://raw.githubusercontent.com/galaxion-tech/AI1110/master/ass_manual/code/2.2.py
\end{lstlisting}

and executed using

\begin{lstlisting}
$ python3 2.2.py
\end{lstlisting}

Graph is as follow:
    \begin{figure}[H]
        \includegraphics[scale=0.6]{./figs/gau_cdf}
        \caption{CDF of X}
    \end{figure}

    CDF has properties:
    \begin{itemize}
        \item CDF is non-decreasing
        \item $\lim_{x \leftarrow -\infty} F_X(x) = 0 $
        \item $ \lim_{x \leftarrow \infty} F_X(x) = 1 $
        \item It is right continous
    \end{itemize}


    \item Load gau.dat in python and plot the empirical PDF of X using the samples in gau.dat. The PDF of X is defined as
    \begin{equation}
        P_X(x) = \frac{d}{dx}F_X(x)
    \end{equation}
    What properties does the PDF have?\\

    \solution 
    The required python file can be downloaded using

    \begin{lstlisting}
 $ wget https://raw.githubusercontent.com/galaxion-tech/AI1110/master/ass_manual/code/2.3.py
\end{lstlisting}

and executed using

\begin{lstlisting}
$ python3 2.3.py
\end{lstlisting}

Graph is as follow:
    \begin{figure}[H]
        \includegraphics[scale=0.6]{./figs/gauss_pdf}
        \caption{PDF of X}
    \end{figure}

    PDF has properties: 
    \begin{itemize}
        \item $\int_{-\infty}^{\infty} P_X(x) dx = 1$
        \item $\forall x \in \mathbb{R} \hspace{12pt} P_X(x) \leq 0$
        \item $\forall a<b \hspace{12pt} a,b \in \mathbb{R} \\ Pr(a<x<b) = Pr(a\leq x\leq b) = \int_a^b P_X(x) dx$
    \end{itemize}


    \item Find the mean and variance of X by writing a C program.\\

    \solution 

\noindent The C program can be downloaded using

\begin{lstlisting}
$ wget https://raw.githubusercontent.com/galaxion-tech/AI1110/master/ass_manual/code/2.4.c
\end{lstlisting}

and compiled and executed with the following commands

\begin{lstlisting}
$ gcc 2.4.c -lm -Wall -g
$ ./a.out
\end{lstlisting} 

On running, we get
    \begin{align}
        E[X] = 0.000326 \\
        \text{Var}[X] = 1.000907
    \end{align}


    \item Given that
     \begin{equation}
        P_X(x) = \frac{1}{\sqrt{2\pi}}\text{exp}\left(\frac{-x^2}{2}\right), -\infty < x < \infty
     \end{equation}
     Find Mean and Varaince theoretically.\\
     \solution we have,
     \begin{align}
        E[X] &= \int_{-\infty}^{\infty}xP_X(x)dx \\
        &= \int_{-\infty}^{\infty}x\frac{1}{\sqrt{2\pi}}e^{\frac{-x^2}{2}}dx \\
        &=\left. \frac{1}{\sqrt{2\pi}}(-e^{\frac{-x^2}{2}}) \right|_{x=-\infty}^{x=\infty}\\
        &=0
     \end{align}
     Now, Knowing the fact $\int_{-\infty}^{\infty} P_x(x) = 1$ \\ Using Integration by Parts, we get,
     \begin{align}
        \text{Var}[X]&=E[X^2]-(E[X])^2 \\
        &=\int_{-\infty}^{\infty}x^2P_x(x)dx \\
        &=\int_{-\infty}^{\infty}x^2\frac{1}{\sqrt{2\pi}}e^{\frac{-x^2}{2}}dx \\
        &=x\int_{-\infty}^{\infty}x\frac{1}{\sqrt{2\pi}}e^{\frac{-x^2}{2}}dx \\ &- \int_{-\infty}^{\infty}\int_{-\infty}^{\infty}x\frac{1}{\sqrt{2\pi}}e^{\frac{-x^2}{2}}dx \\
        &=\left. x.[-\frac{1}{\sqrt{2\pi}}e^{\frac{-x^2}{2}}]\right|_{x=-\infty}^{x=\infty} \\
        &+ \int_{-\infty}^{\infty}\frac{1}{\sqrt{2\pi}}e^{\frac{-x^2}{2}}dx \\
        &=0+\int_{-\infty}^{\infty}\frac{1}{\sqrt{2\pi}}e^{\frac{-x^2}{2}}dx\\
        &=\int_{-\infty}^{\infty}\frac{1}{\sqrt{2\pi}}e^{\frac{-x^2}{2}}dx\\
        &=\int_{-\infty}^{\infty}P_X(x)dx\\
        &=1
     \end{align}
\end{enumerate}
\section{From Uniform to other}
\begin{enumerate}[label=\thesection.\arabic*.,ref=\thesection.\theenumi]
\numberwithin{equation}{enumi}
\numberwithin{figure}{enumi}
\numberwithin{table}{enumi}

\item Generate samples of
\begin{equation}
    V = -2 \ln (1-U)
\end{equation}
and plot its CDF.\\
\solution 

Download the C code to create the distribution.

\begin{lstlisting}
$ wget https://raw.githubusercontent.com/galaxion-tech/AI1110/master/ass_manual/code/header/coeffs.h
$ wget https://raw.githubusercontent.com/galaxion-tech/AI1110/master/ass_manual/code/3.1.c
\end{lstlisting}
and can be executed with
\begin{lstlisting}
$ gcc 3.1.c -lm -Wall -g
$ ./a.out
\end{lstlisting}

The relevant python code is at

\begin{lstlisting}
 $ wget https://raw.githubusercontent.com/galaxion-tech/AI1110/master/ass_manual/code/3.1.py
\end{lstlisting}
and can be executed with
\begin{lstlisting}
$ python3 3.1.py
\end{lstlisting}

CDF Graph is as follow 
\begin{figure}[H]
    \includegraphics[scale=0.6]{./figs/other_cdf}
    \caption{CDF of V}
\end{figure}

\item Find a theoretical expression for $F_V (x)$.\\
\solution 
    Since $V=f(U) = -2\ln(1-U)$ is a monotonically incresing function in $(0,\infty)$. Therefore, It's Inverse exists:
    \\ $U=f^{-1}(V) = 1-e^{-v/2}$\\
    Hence By monotonicity of f(U), we get
    \begin{align}
        F_V(x) &= Pr(V<x)\\
        &= Pr(-2\ln(1-U) < x)\\
        &= Pr(U<1-e^{\frac{-x}{2}}) \\
        &= F_U(1-e^{\frac{-x}{2}})
    \end{align}
    Therfore,
    \begin{align}
        F_V(x) = \begin{cases}
        0 & x \in (-\infty,0] \\
        1-e^{\frac{-x}{2}} & x \in (0,\infty)
    \end{cases} 
    \end{align}
\end{enumerate}

\section{Triangular Distribution }
\begin{enumerate}[label=\thesection.\arabic*.,ref=\thesection.\theenumi]
    \numberwithin{equation}{enumi}
    \numberwithin{figure}{enumi}
    \numberwithin{table}{enumi}
    
    \item Generate 
    \begin{align}
        T=U_1+U_2
    \end{align} 
    \solution Download the files:
    \begin{lstlisting}
$ wget https://raw.githubusercontent.com/galaxion-tech/AI1110/master/ass_manual/code/4.1.c
$ wget https://raw.githubusercontent.com/galaxion-tech/AI1110/master/ass_manual/code/header/coeffs.h
    \end{lstlisting}
    and compile and execute the C program using
    \begin{lstlisting}
$ gcc 4.1.c -lm -Wall -g
$ ./a.out
    \end{lstlisting}
    
    \item Find the CDF of $T$.\\
    \solution 
    The following code plots \ref{fig 4.2}
    \begin{lstlisting}
$ wget https://raw.githubusercontent.com/galaxion-tech/AI1110/master/ass_manual/code/4.2.py
    \end{lstlisting}
    and execute with
    \begin{lstlisting}
$ python3 4.2.py
    \end{lstlisting}
    Experimental graph of CDF is as follow:
    \begin{figure}[H]
        \includegraphics[scale=0.6]{./figs/4.2}
        \label{fig 4.2}
        \caption{Experimental CDF of T}
    \end{figure}
    \item Find the PDF of $T$.\\
    \solution 
    The following code plots \ref{fig 4.3}
    \begin{lstlisting}
$ wget https://raw.githubusercontent.com/galaxion-tech/AI1110/master/ass_manual/code/4.3.py    
    \end{lstlisting}
    and execute with
    \begin{lstlisting}
$ python3 4.3.py 
    \end{lstlisting}
    Experimental graph of PDF is as follow:
    \begin{figure}[H]
        \includegraphics[scale=0.6]{./figs/4.3}
        \label{fig 4.3}
        \caption{Experimental PDF of T}
    \end{figure}
    \item Find the theoretical expressions for the PDF and CDF of T.\\
    \solution we have
    \begin{equation}
        T=U_1+U_2
    \end{equation}
    we know,
    \begin{align}
        p_U(x) = \begin{cases}
        1 & x \in (0,1) \\
        0 & otherwise
    \end{cases} 
    \end{align}
    By convolution, we have
    \begin{align}
        p_T(x)&=p_U(u)*p_U(u) \\
        &=\int_{-\infty}^{\infty} p_U(\tau)p_U(x-\tau)d\tau \\
    \end{align}
    Since $p_U(\tau)$ is 0 when $x < -\infty$ and $x>1$ \\
    Therefore,
    \begin{align}
        \int_{-\infty}^{\infty} p_U(\tau)p_U(x-\tau)d\tau &= \int_{0}^{1} p_U(\tau)p_U(x-\tau)d\tau \\
        &=\int_{0}^{1}p_U(x-\tau)d\tau
    \end{align}
    Now When $0<x<1$ \\
    \begin{align}
        \int_{0}^{1} p_U(x-\tau)d\tau &= \int_{0}^{x}p_U(x-\tau)d\tau \\
        &= \int_{0}^{x}1d\tau \\
        &= x
    \end{align}
    Now, When $1<x<2$ \\
    \begin{align}
        \int_{0}^{1} p_U(x-\tau)d\tau &= \int_{1-x}^{1}p_U(x-\tau)d\tau \\
        &= \int_{1-x}^{1}1d\tau \\
        &= 2-x
    \end{align}
    Therefore,
    \begin{align}
        p_T(x)=\begin{cases}
            x & x \in (0,1] \\
            2-x & x \in (1,2)
        \end{cases}
    \end{align}
    we know,
    \begin{equation}
        F_T(x) = \int_{-\infty}^{x}P_T(t)dt
       \end{equation}
    Therefore,
    \begin{align}
        F_T(x)=\begin{cases}
            0 & x \in (-\infty,0) \\
            x^2/2 & x \in (0,1] \\
            -x^2/2+2x-1 & x \in (1,2) \\
            1 & x \in [2,\infty)
        \end{cases}
    \end{align}

    \item Verify your results through a plot. \\
    \solution 
    The following code plots \ref{fig 4.5}
    \begin{lstlisting}
$ wget https://raw.githubusercontent.com/galaxion-tech/AI1110/master/ass_manual/code/4.5.py
    \end{lstlisting}
    and execute with
    \begin{lstlisting}
$ python3 4.5.py
    \end{lstlisting}

    The follwing code plots \ref{fig 4.6}
    \begin{lstlisting}
$ wget https://raw.githubusercontent.com/galaxion-tech/AI1110/master/ass_manual/code/4.6.py 
    \end{lstlisting}
    and execute with
    \begin{lstlisting}
$ python3 4.6.py
    \end{lstlisting}
    Graph of CDF is as follow:
    \begin{figure}[H]
        \includegraphics[scale=0.6]{./figs/4.5}
        \caption{CDF of T}
        \label{fig 4.5}
    \end{figure}
    Graph of PDF is as follow:
    \begin{figure}[H]
        \includegraphics[scale=0.6]{./figs/4.6}
        \label{fig 4.6}
        \caption{PDF of T}
    \end{figure}
\end{enumerate}

\section{Maximum Likelihood}
\begin{enumerate}[label=\thesection.\arabic*.,ref=\thesection.\theenumi]
\numberwithin{equation}{enumi}
\numberwithin{figure}{enumi}
\numberwithin{table}{enumi}

\item Generate equiprobable $X \in {-1,1}$. \\
\solution Download the file\\
\begin{lstlisting}
$ wget https://raw.githubusercontent.com//galaxion-tech/AI110/master/ass_manual/code/5.1.c
\end{lstlisting}
Use the coeffs.h downloaded in problem 1.1\\
Execute the code as follow:
\begin{lstlisting}
$ gcc ./5.1.c -Wall -g -lm
$ ./a.out
\end{lstlisting}
\item Generate \begin{align}
    Y=AX+N
\end{align}
where $A=5$ dB, and $N \sim N(0,1)$.\\
\solution Download the file\\
\begin{lstlisting}
$ wget https://raw.githubusercontent.com//galaxion-tech/AI110/master/ass_manual/code/5.2.c
\end{lstlisting}
Use the coeffs.h downloaded in problem 1.1\\
Execute the code as follow:
\begin{lstlisting}
$ gcc ./5.2.c -Wall -g -lm
$ ./a.out
\end{lstlisting}
\item Plot $Y$ using a scatter plot.\\
\solution Download the Python code
\begin{lstlisting}
$ wget https://raw.githubusercontent.com//galaxion-tech/AI110/master/ass_manual/code/5.3.py
\end{lstlisting}
\begin{lstlisting}
$ python3 ./5.3.py
\end{lstlisting}
Noise Produced as follow:
\begin{figure}[H]
    \includegraphics[scale=0.6]{./figs/5.3.png}
    \label{fig 5.3}
    \caption{Noise of Y}
\end{figure}
\item Guess how to estimate $X$ from $Y$\\
\solution we have\\
\begin{align}
    Y=AX+N
\end{align}
Estimating X from Y as follows:
\begin{align}
   \hat{X} = sgn(Y) 
\end{align}
where $sgn(y)$ is defined as 
\begin{align}
    sgn(y)=\begin{cases}
        -1 & y \in (-\infty,0) \\
        -1 & y \in [0,\infty)
    \end{cases}
\end{align}
\item Find \begin{align}
    P_{e|0} = Pr(\hat{X}=-1|X=1)
\end{align}
and
\begin{align}
    P_{e|1} = Pr(\hat{X}=1|X=-1)
\end{align}
\solution Download the code
\begin{lstlisting}
$ wget https://raw.githubusercontent.com//galaxion-tech/AI110/master/ass_manual/code/5.5.c
\end{lstlisting}
Execute it
\begin{lstlisting}
$ gcc 5.5.c -lm -Wall -g
$ ./a.out
\end{lstlisting}
Now Download the python code
\begin{lstlisting}
$ wget https://raw.githubusercontent.com//galaxion-tech/AI110/master/ass_manual/code/5.5.py
\end{lstlisting}
Execute it
\begin{lstlisting}
$ python3 5.5.py
\end{lstlisting}
On executing, we get
\begin{align}
    P_{e|0} &=0.3100037999240015 \\
    P_{e|1} &=0.3106582131642633
\end{align}
\item  Find $P_e$ assuming that $X$ has equiprobable symbols.\\
\solution we get
\begin{align}
    P_e &=\frac{P_{e|0}+P_{e|1}}{2} \\
    &=\frac{0.3106582131642633+0.3106582131642633}{2} \\ &= 0.31033100654413237
\end{align}
\item Verify by plotting the theoretical $P_e$ with respect to $A$ from 0 to 10 dB.\\
\solution Download the file
\begin{lstlisting}
$ wget https://raw.githubusercontent.com//galaxion-tech/AI110/master/ass_manual/code/5.7.c
\end{lstlisting} 
Use the coeffs.h downloaded in problem 1.1\\
Now compile it\\
\begin{lstlisting}
$ gcc ./5.7.c -o 5.7 -lm -Wall -g
\end{lstlisting}
Now Download the Python code
\begin{lstlisting}
$ wget https://raw.githubusercontent.com//galaxion-tech/AI110/master/ass_manual/code/5.7.py
\end{lstlisting}
and Execute It
\begin{lstlisting}
$ python3 5.7.py
\end{lstlisting}
On executing, we have
\begin{figure}[H]
    \includegraphics[scale=0.6]{./figs/5.7.png}
    \label{fig 5.7}
    \caption{}
\end{figure}
Theoretical Explanation:\\
For equiprobable $X$, we have
\begin{align}
    P_e = P_{e|0} = P_{e|1} \label{5.7.1}
\end{align}
Now,
\begin{align}
    P_e&=P_{e|0} \\
    &=Pr(\hat{X}=-1|X=1) \\
    &=Pr(sgn(Y)=-1|X=1) \\ 
    &= Pr(Y<0|X=1)\\
    &=Pr(AX+N<0|X=1) \\ 
    &= Pr(A+N < 0)\\
    &=Pr(N<-A) \\ 
    &= 1-Pr(N<A) \\
    &=1-F_N(A)\\
    &=Q_N(A) \label{5.7.11}
\end{align}
Hence Shown 
\item Now consider a threshold $\delta$ while estimating $X$ from $Y$. Find the value of $\delta$ that minimize the theoretical $P_e$.\\
\solution Let estimation of X from Y have an threshold $\delta$
\begin{align}
    \hat{X}=\begin{cases}
        1 & Y> \delta \\
        -1 & Y < \delta
    \end{cases}
\end{align}
Since we know,
\begin{align}
    P_{e|0} &= Pr(\hat{X}=-1|X=1)\\
    &=Pr(Y<\delta | X=1)\\
    &=Pr(A+N<\delta)\\
    &=Pr(N<\delta-A)\\
    &=F_N(\delta-A)\label{5.8.6}\\
    P_{e|1} &= Pr(\hat{X}=1|X=-1)\\
    &=Pr(Y>\delta | X=-1)\\
    &=Pr(-A+N>\delta)\\
    &=Pr(N>\delta+A)\\
    &=Q_N(\delta+A) \label{5.8.11}
\end{align}
Therefore,
\begin{align}
    P_e &=\frac{P_{e|0}+P_{e|1}}{2}\\
    &=\frac{F_N(\delta-A)+Q_N(\delta+A)}{2}
\end{align}
For minimizing, differnciating w.r.t $\delta$
\begin{align}
    \frac{d P_e}{d\delta} &= \frac{d}{d\delta}\left(\frac{F_N(\delta-A)+Q_N(\delta+A)}{2}\right)\\
    &=\frac{P_N(\delta-A)-P_N(\delta+A)}{2}\\
    &=\frac{1}{2\sqrt{2\pi}}\left[e^{\frac{-(\delta-A)^2}{2}}-e^{\frac{-(\delta+A)^2}{2}}\right]\\
    &=0
\end{align}
Only Possible solution for $\delta $ are
\begin{align}
    \delta=0,\pm \infty
\end{align}
Since for minima \\
\begin{align}
    \frac{d^2P_e}{d\delta ^2} >0
\end{align}
On calculating double derivative
\begin{align}
    \frac{d^2P_e}{d\delta ^2} &= \frac{1}{2\sqrt{2\pi}}[(A-\delta)e^{\frac{-(\delta-A)^2}{2}}\\
     &+(A+\delta)e^{\frac{-(\delta+A)^2}{2}}]
\end{align}
when $\delta =0$, we get
\begin{align}
    \frac{d^2P_e}{d\delta ^2} &= \frac{1}{2\sqrt{2\pi}}[(A)e^{\frac{-(A)^2}{2}}+(A)e^{\frac{-(A)^2}{2}}] \\
    &>0
\end{align}
Hence $\delta=0$ is the threshold on which $P_e$ minimized.
\item Repeat the above exercise when
\begin{align} p_X(0) = p \end{align}
\solution Now we have 
\begin{align}
    p_X(x) = \begin{cases}
        p & x = 1 \\
        1-p & x= -1
    \end{cases}
\end{align}
Since from \eqref{5.7.1} and \eqref{5.7.11}
we have
\begin{align}
    P_{e|0}=P_{e|1} = Q_N(A)
\end{align}
Therfore,\\
\begin{align}
    P_e &= (p)P_{e|0}+(1-p)P_{e|1} \\
    &= (p)Q_N(A)+(1-p)Q_N(A)\\
    &=(p+1-p)Q_N(A)\\
    &=Q_N(A)
\end{align}
Hence $P_e$ is independent of $p$ when threshold $\delta=0$\\
Now Consider a threshold $\delta$\\
From \eqref{5.8.6} and \eqref{5.8.11}, we have
\begin{align}
    P_e &=(p)p_{e|0}+(1-p)p_{e|1}\\
    &= (p)F_N(\delta-A)+(1-p)Q_N(\delta+A)
\end{align}
On differnciating, we have\\
\begin{align}
    \frac{dP_e}{d\delta} &= (p)P_N(\delta-A)-(1-p)P_N(\delta+A) \\
    &=\frac{1}{\sqrt{2\pi}}\left[(p)e^{\frac{-(\delta-A)^2}{2}}-(1-p)e^{\frac{-(\delta+A)^2}{2}}\right] \\
    &=0
\end{align}
On calculating
\begin{align}
   \left[(p)e^{\frac{-(\delta-A)^2}{2}}-(1-p)e^{\frac{-(\delta+A)^2}{2}}\right]&=0 \\
   e^{\frac{(\delta+A)^2}{2}-\frac{(\delta-A)^2}{2}}&=\frac{1}{p} -1\\
   e^{2\delta A} &= \frac{1}{p} -1
\end{align}
Since $e^x$ is monotonic.
\begin{align}
   \delta &= \frac{1}{2A}\ln (\frac{1}{p}-1)
\end{align}
On double differnciating $P_e$ at $\delta$ =$\frac{1}{2A}\ln (\frac{1}{p}-1)$ for minima, we get
\begin{align}
    \frac{d^2P_e}{d\delta ^2} &= \frac{1}{2\sqrt{2\pi}}[(p)(A-\delta)e^{\frac{-(\delta-A)^2}{2}}\\ 
    &+(1-p)(A+\delta)e^{\frac{-(\delta+A)^2}{2}}] \\
    & >0
\end{align}
Now,
\begin{align}
    p(A-\delta)e^{\frac{-(\delta-A)^2}{2}} &> -(1-p)(\delta +A)e^{\frac{-(\delta+A)^2}{2}}\\
    p(A-\delta)e^{2\delta \pi} &> -(1-p)(A+\delta)\\
    (A-\delta)(1-p) &> (p-1)(A+\delta)\\
    A-Ap &> Ap-A\\
    2A &> 2Ap \hspace*{12pt} \text{Since} A \neq 0 \\
    p&<1
\end{align}
Hence Always true\\
Therefore, $\delta$ = $\frac{1}{2A}\ln (\frac{1}{p}-1)$ is the threshold where $P_e$ minimized given $p_X(0)=1$\\

\item Repeat the above exercise using the MAP criterion.\\
\solution 
\end{enumerate}


\section{Gaussian To Other}
\begin{enumerate}[label=\thesection.\arabic*.,ref=\thesection.\theenumi]
\numberwithin{equation}{enumi}
\numberwithin{figure}{enumi}
\numberwithin{table}{enumi}

\item Let $X_1 \sim N(0,1)$ and $X_2 \sim N(0,1)$. Plot the CDF and PDF of 
\begin{align}
    V=X_1^2+X_2^2
\end{align}
\solution Downlaod the C code to generate the distribution \\
\begin{lstlisting}
$ wget https://raw.githubusercontent.com/galaxion-tech/AI1!10/master/ass_manual/code/6.1.c
\end{lstlisting}
Use the coeffs.h downloaded in problem 1.1
Now execute it to get distribution in v.dat file
\begin{lstlisting}
$ gcc 6.1.c -lm -Wall -g
$ ./a.out
\end{lstlisting}
Now Download the python code 
\begin{lstlisting}
$ wget https://raw.githubusercontent.com/galaxion-tech/AI1!10/master/ass_manual/code/6.1_cdf.py
$ wget https://raw.githubusercontent.com/galaxion-tech/AI1!10/master/ass_manual/code/6.1_pdf.py
\end{lstlisting}
execute them
\begin{lstlisting}
$ python3 6.1_cdf.py
$ python3 6.1_pdf.py
\end{lstlisting}

\begin{figure}[H]
    \includegraphics[scale = 0.6]{./figs/6.1_cdf.png}
    \label{6.1.1}
    \caption{CDF of V}
\end{figure}
\begin{figure}[H]
    \includegraphics[scale = 0.6]{./figs/6.1_pdf.png}
    \label{6.1.2}
    \caption{PDF of V}
\end{figure}
Theoritical Explanation:
\begin{align}
    V=X_1^2+X_2^2
\end{align}
where $X_1$ and $X_2$ are i.i.d normal random variable \\
Consider two random variable $R$ and $\Theta$ such that $X_1 = R \sin \Theta$ and $X_2 = R \cos \Theta$,\\
Using transformation, we have
\begin{align}
    f_{R,\Theta}(r,\theta) = \|J\| f_{X_1,X_2}(x_1,x_2) \label{6.1.3}
\end{align}
where $J$ is Jacobian matrix
\begin{align}
    J &= \begin{pmatrix}
        \frac{\delta x_1}{\delta r} & \frac{\delta x_1}{\delta \theta} \\
        \frac{\delta x_2}{\delta r} & \frac{\delta x_2}{\delta \theta}
    \end{pmatrix} \\
    &=\begin{pmatrix}
        \sin \theta & r \cos \theta \\
        \cos \theta & -r \sin \theta 
    \end{pmatrix}\\
    &= -r(\cos^2 \theta + \sin^2 \theta) \\
    \|J\|&= r
\end{align}
Now, since $X_1$ and $X_2$ are independent,
\begin{align}
    f_{X_1,X_2}(x_1,x_2) &= f_{X_1}(x_1).f_{X_2}(x_2) \\
    &= \left(\frac{1}{\sqrt{2\pi}}e^{\frac{-x_1^2}{2}}\right) \left(\frac{1}{\sqrt{2\pi}}e^{\frac{-x_2^2}{2}}\right) \\
    &= \left(\frac{1}{2\pi}e^{\frac{-(x_1^2+x_2^2)}{2}}\right)
\end{align} 
Now Since $X_1^2+X_2^2 = R^2$, using \eqref{6.1.3} we get\\
\begin{align}
    f_{R,\Theta}(r,\theta) = \frac{r}{2\pi} e^{\frac{-r^2}{2}} 
\end{align}
Now we know $R$ and $\Theta$ are independent, Therefore
\begin{align}
    f_R(r) &= \int_{0}^{2\pi} f_{R,\Theta}(r,\theta) d\theta \\
    &= \int_{0}^{2\pi} \frac{r}{2\pi} e^{\frac{-r^2}{2}} d\theta \\
    &= r e^{\frac{-r^2}{2}} \label{6.1.14}
\end{align}
Now Since $V=X_1^2+X_2^2 = R^2$ where $R \geq 0$, we have CDF of V as follow
\begin{align}
    F_V(v) &= Pr(V < v) \\
    &= Pr(R^2 < v) \\
    &=Pr(R < \sqrt{v}) \hspace*{12pt} \text{Since } R\geq 0\\
    &=F_R(\sqrt{v}) \label{6.1.18}
\end{align}
where $v \geq 0$\\
Since when $v< 0$,From \eqref{6.1.18} we have 
\begin{align}F_V(v) = 0\end{align}
Therefore,\begin{align}
    f_V(v) = 0
\end{align}
when $v \geq 0$, \\
On differnciating both side w.r.t $v$, we get,
\begin{align}
    F_V(v) &= F_R(\sqrt{v})\\
    \frac{dF_V(v)}{dv} &= \frac{dF_R(\sqrt{v})}{dv} \\
    f_V(v) &= f_R(\sqrt{v})\frac{1}{2\sqrt{v}} \\
    &=\sqrt{v}e^{\frac{-v}{2}}\frac{1}{2\sqrt{v}}\\
    &=\frac{1}{2}e^{\frac{-v}{2}}
\end{align}

Now, On Integrating $f_V(v)$ for $v \geq 0$, we get
\begin{align}
    F_V(v) &= \int_{-\infty}^{v} f_V(v) dv \\
    &= \int_{-\infty}^{v} \frac{1}{2}e^{\frac{-v}{2}} dv \\
    &= 1-e^{\frac{-v}{2}}
\end{align}
Hence we have
\begin{align}
    f_V(v) &= (\frac{1}{2}e^{\frac{-v}{2}})u(v)\\
    F_V(v) &= (1-e^{\frac{-v}{2}})u(v) \label{6.1.30}
\end{align}
where $u(v)$ is a unit step function\\
It is Chi-square Distribution.
\item If 
\begin{align}
    F_V(x) = \begin{cases}
        1-e^{-\alpha x} & x \geq 0\\
        0 & x < 0
    \end{cases} \label{6.2.1}
\end{align}
find $\alpha$.\\
\solution Since From \eqref{6.1.30}, we have
\begin{align}
    F_V(x) = \begin{cases}
        1-e^{\frac{-x}{2}} & x \geq 0 \\
        0 & x <0
    \end{cases} \label{6.2.2}
\end{align}
Since $e^{-x}$ is monotonic, On comparing \eqref{6.2.2} with \eqref{6.2.1}, we get
\begin{align}
    \alpha =\frac{1}{2}
\end{align}
\item Plot the CDF and PDF of \begin{align}
    A=\sqrt{V}
\end{align}
\solution Download the C code to generate distribution A in aa.dat
\begin{lstlisting}
$ wget http://raw.githubusercontent.com/galaxion-tech/AI1110/master/ass_manual/code/6.3.c
\end{lstlisting}
execute it to get distribution
\begin{lstlisting}
$ gcc 6.3.c -lm -Wall -g
$ ./a.out
\end{lstlisting}
Now Downlaod the Python code 
\begin{lstlisting}
$ wget https://raw.githubusercontent.com/galaxion-tech/AI1110/master/ass_manual/code/6.3_cdf.py
$ wget https://raw.githubusercontent.com/galaxion-tech/AI1110/master/ass_manual/code/6.3_pdf.py
\end{lstlisting}
Execute it 
\begin{lstlisting}
$ python3 6.3_cdf.py
$ python3 6.3_pdf.py
\end{lstlisting}
\begin{figure}[H]
    \includegraphics[scale = 0.6]{./figs/6.3_cdf.png}
    \label{6.3_cdf}
    \caption{CDF of A}
\end{figure}
\begin{figure}[H]
    \includegraphics[scale = 0.6]{./figs/6.3_pdf.png}
    \label{6.3_pdf}
    \caption{PDF of A}
\end{figure}
Theoritical Explanation:\\
Since we have $V=R^2$, Hence
\begin{align}
    A = \sqrt{V} = R
\end{align}
Therefore,From \eqref{6.1.14}, PDf of A is as follow
\begin{align}
    f_A(a) &= f_R(a) \\
    &= ae^{\frac{-a^2}{2}} \label{6.3.4}
\end{align}
where $a \geq 0$\\
For $a<0$, we have
\begin{align}
    f_A(a) &= 0\\
    F_A(a) &= 0
\end{align}
On Integrating \eqref{6.3.4} for $a \geq 0$, we have 
\begin{align}
    F_A(a)&=\int_{-infty}^{a} f_A(x)dx \\
    &=\int_{-\infty}^{a} ae^{\frac{-a^2}{2}}\\
    &=\int_{-\infty}^{0} ae^{\frac{-a^2}{2}}+\int_{0}^{a} ae^{\frac{-a^2}{2}}\\
    &=0+\int_{0}^a ae^{\frac{-a^2}{2}}\\
    &=1-e^{\frac{-a^2}{2}}
\end{align}
Therefore, Finally we have
\begin{align}
    f_A(a) &= ae^{\frac{-a^2}{2}}u(a) \\
    F_A(a) &= (1-e^{\frac{-a^2}{2}})u(a)
\end{align}
where $u(a)$ is unit step function\\
It is Rayleigh Distribution
\end{enumerate}
\section{Conditional Probability}
\begin{enumerate}[label=\thesection.\arabic*.,ref=\thesection.\theenumi]
\numberwithin{equation}{enumi}
\numberwithin{figure}{enumi}
\numberwithin{table}{enumi}
\item Plot \begin{align}
    P_e = Pr(\hat{X} = -1 | X=1) \label{7.1}
\end{align}
for \begin{align}
    Y=AX+N
\end{align}
where A is Raleigh with $E[A^2]=\gamma,N \sim N(0,1), X \in \{-1,1\}$ for $0 \leq \gamma \leq 10$ dB. 
\item Assuming that N is a constant, find an expression for $P_e$. Call this $P_e(N)$
\item For a function $g$,
\begin{align}
    E[g(X)] = \int_{-\infty}^{\infty} g(x)p_X(x) dx \label{7.3}
\end{align}
Find $P_e = E[P_e(N)]$.
\item Plot $P_e$ in problem \eqref{7.1} and \eqref{7.3} on the same graph w.r.t $\gamma$. Comment.
\end{enumerate}
\end{document}