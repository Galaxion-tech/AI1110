%%%%%%%%%%%%%%%%%%%%%%%%%%%%%%%%%%%%%%%%%%%%%%%%%%%%%%%%%%%%%%%
%
% Welcome to Overleaf --- just edit your LaTeX on the left,
% and we'll compile it for you on the right. If you open the
% 'Share' menu, you can invite other users to edit at the same
% time. See www.overleaf.com/learn for more info. Enjoy!
%
%%%%%%%%%%%%%%%%%%%%%%%%%%%%%%%%%%%%%%%%%%%%%%%%%%%%%%%%%%%%%%%


% Inbuilt themes in beamer
\documentclass{beamer}

% Theme choice:
\usetheme{CambridgeUS}

% Title page details: 
\title{Assignment 10} 
\author{CS21BTECH11020 (Harsh Goyal)}
\date{\today}
\logo{\large \LaTeX{}}


\begin{document}

% Title page frame
\begin{frame}
    \titlepage 
\end{frame}

% Remove logo from the next slides
\logo{}


% Outline frame
\begin{frame}{Outline}
    \tableofcontents
\end{frame}


% Lists frame
\section{Problem Statement}
\begin{frame}{Problem Statement}
\begin{block}{Papoulis Ch-9 Ex-9.37 }
The process $x(t)$ is normal with zero mean and $R_x(\tau)$ = $Ie^{-\alpha |\tau|}cos(\beta \tau)$. Show that if $y(t)$ =
$x^2(t)$. then $C_y(\tau)$ = $I^2e^{-2 \alpha |\tau|}(1 + cos(2\beta\tau))$. Find $S_y(\omega)$.
    
\end{block}

\end{frame}


% Blocks frame
\section{Solution}
\begin{frame}{Solution}
 \begin{block}{}
 We Know,
 \begin{equation}
     E\{x^2(t+\tau)x^2(t)\} = E\{x^2(t+\tau)\}E\{x^2(t)\} + 2E^2\{x^2(t+\tau)x^2(t)\}
 \end{equation}
 Hence,
 \begin{equation}
     R_y(\tau) = R_x^2(0)+2R_x^2(\tau) = I^2(1+e^{-2\alpha |\tau|}+e^{-2\alpha |\tau|}cos(2\beta \tau))
 \end{equation}
 \begin{equation}
     S_y(\omega) = \left[ 2\pi\delta(\omega) + \frac{4\alpha}{4\alpha^2+\omega^2}+\frac{2\alpha}{4\alpha^2+(\omega-2\beta)^2}+ \frac{2\alpha}{4\alpha^2+(\omega+2\beta)^2} \right]
 \end{equation}
 Futhermore,
 \begin{align}
	 n_y&= E\{x^2(t)\} = R_x(0) \\
	 C_y(\tau)&= 2R_x^2(\tau)
 \end{align}
 \end{block}
\end{frame} 

\end{document}

