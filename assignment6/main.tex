%%%%%%%%%%%%%%%%%%%%%%%%%%%%%%%%%%%%%%%%%%%%%%%%%%%%%%%%%%%%%%%
%
% Welcome to Overleaf --- just edit your LaTeX on the left,
% and we'll compile it for you on the right. If you open the
% 'Share' menu, you can invite other users to edit at the same
% time. See www.overleaf.com/learn for more info. Enjoy!
%
%%%%%%%%%%%%%%%%%%%%%%%%%%%%%%%%%%%%%%%%%%%%%%%%%%%%%%%%%%%%%%%


% Inbuilt themes in beamer
\documentclass{beamer}

% Theme choice:
\usetheme{CambridgeUS}

% Title page details: 
\title{ASSIGNMENT 6} 
\author{HARSH GOYAL (CS21BTECH11020)}
\date{\today}
\logo{\large \LaTeX{}}

\usepackage{amssymb}

\begin{document}

% Title page frame
\begin{frame}
    \titlepage 
\end{frame}

% Remove logo from the next slides
\logo{}


% Outline frame
\begin{frame}{Outline}
    \tableofcontents
\end{frame}


% Problem
\section{Problem Statement}
\begin{frame}{Problem Statement}
    \begin{block}{Papoulis chap 2 Ex 2.12}
        A call occurs at time $t$, where $t$ is a random point in the interval (0,10).
        \begin{itemize}
            \item Find $P(6 \leq t \leq 8)$
            \item Find $P(6 \leq t \leq 8 | t > 5 )$
        \end{itemize}
              
    \end{block}

\end{frame}


% Solution
\section{Solution}
\begin{frame}{Solution}
    \begin{alertblock}{Bijection}
        We have bijection between n(0,1) and n(a,a+1). Let assume a function $f:(0,1) \Rightarrow (a,a+1)$ as $f(x) = a+x$ \\
        Following function is one-one and onto. Therefore it is a bijective function.\\
        Hence we have 
        \begin{equation}
            \| (0,1) \| = \| (a,a+1) \| \implies n(0,1) = n(a,a+1) \label{eq:1}
        \end{equation}
    \end{alertblock}
   \begin{block}{Part 1}
    Let n(a,b) define the number of real points between a and b in real number line.
    Using equation \eqref{eq:1} ,we get
        \begin{equation}
            P(6 \leq t \leq 8) = \frac{n(6,8)}{n(0,10)} = \frac{2\times n(0,1)}{10 \times n(0,1)} = \frac{2}{10} = 0.2 \label{eq:2}
        \end{equation}
   \end{block}

\end{frame} 
\begin{frame}{Solution}
    \begin{alertblock}{Conditional Probability}
        The conditional probability of an event A assuming another event M, denoted by $P(A|M)$, is by definition the ratio
        \begin{equation}
            P(A|M) = \frac{P(AM)}{P(M)} \label{eq:3}
        \end{equation}
        where we assume that $P(M) \neq 0$.\\
        Now, If  $A \subseteq B$ then
        \begin{equation}
            P(A|M) = \frac{P(A)}{P(M)} \label{eq:4}
        \end{equation}
    \end{alertblock}
    \begin{block}{Part 2}
        Now, Let A = Event of choosing number between 6 and 8 and M = Event of choosing number greater than 5
    \end{block}
\end{frame}

\begin{frame}{Soltuion}
    \begin{block}{Continued \ldots}
        Since $A \subset M$, Using Equation \eqref{eq:3} and \eqref{eq:4}, we have
        \begin{equation}
            P(A|M) = \frac{P(AM)}{P(M)} = \frac{P(A)}{P(M)} = \frac{P(6 \leq t \leq 8)}{P(t > 5)} 
        \end{equation}
        we have,
        \begin{equation}
            P(t > 5) = \frac{n(5,10)}{n(0,10)} = \frac{5 \times n(0,1)}{10 \times n(0,1)} = 0.5 \label{eq:6}
        \end{equation}
        Therefore, Using equation \eqref{eq:2} and \eqref{eq:6}, we get
        \begin{equation}
            P(A|M) = \frac{P(6 \leq t \leq 8)}{P(t > 5)} = \frac{0.2}{0.5} = 0.4
        \end{equation}
    \end{block}
    
\end{frame}

\end{document}
